% This is the aspauthor.tex LaTeX file
% Copyright 2010, Astronomical Society of the Pacific Conference Series

\documentclass[11pt,twoside]{article}
\usepackage{asp2010}

\resetcounters

\bibliographystyle{asp2010}

\markboth{Thomas et al.}{FITS and the needs of modern astronomical research}

\begin{document}

\title{Why FITS is failing to meet the needs of modern astronomical
  research}
\author{Brian~Thomas$^1$, Frossie~Economou$^1$, Perry~Greenfield$^2$,
Paul~Hirst$^3$, Tim~Jenness$^4$, David~S.~Berry$^5$, Erik~Bray$^2$,
Norman~Gray$^6$, Demitri~Muna$^7$, James~Turner$^8$,
Miguel~de~Val-Borro$^9$, Jaunde~Santander~Vela$^{10}$,
David~Shupe$^{11}$, and John~Good$^{12}$
\affil{$^1$National Optical Astronomy Observatory, 950 N.\ Cherry Ave,
Tucson, AZ 85719, USA}
\affil{$^2$Space Telescope Science Institute, 3700 San Martin Drive,
  Baltimore, MD 21218, USA}
\affil{$^3$Gemini Observatory, 670 N.\ A`oh\=ok\=u Place, Hilo, HI
  96720, USA}
\affil{$^4$Department of Astronomy, Cornell University, Ithaca NY,
  14853, USA}
\affil{$^5$Joint Astronomy Centre, 660 N.\ A`oh\=ok\=u Place, Hilo, HI
96720, USA}
\affil{$^6$School of Physics \& Astronomy, University of Glasgow,
  Glasgow, G12~8QQ, United Kingdom}
\affil{$^7$Department of Astronomy, Ohio State University, Columbus,
  OH 43210, USA}
\affil{$^8$Gemini Observatory, Casilla 603, La Serena, Chile}
\affil{$^9$Department of Astrophysical Sciences, Princeton University,
Princeton, NY 08544, USA}
\affil{$^{10}$Instituto de Astrof\'{i}sica de Analuc\'{i}a, Glorieta
  de la Astronom\'{i}a s/n, E-18008, Granada, Spain}
\affil{$^{11}$Infrared Processing and Analysis Center, Caltech,
  Pasadena, CA 91125, USA}
}

\begin{abstract}
  The Flexible Image Transport System (FITS) standard has been a great
  boon to astronomy, allowing observatories, scientists and the public
  to exchange astronomical information easily. The FITS standard is,
  however, showing its age. Developed in the late 1970's the FITS
  authors made a number of implementation choices for the format that,
  while common at time, are now seen to limit its utility with modern
  data. The authors of the FITS standard could not appreciate the
  challenges which we would be facing today in astronomical
  computing. Difficulties we now face include, but are not limited to,
  having to address the need to handle an expanded range of
  specialized data product types (data models), being more conducive
  to the networked exchange and storage of data, handling very large
  datasets and the need to capture significantly more complex metadata
  and data relationships.

  There are members of the community today who find some (or all) of
  these limitations unworkable, and have decided to move ahead with
  storing data in other formats. This reaction should be taken as a
  wakeup call to the FITS community to make changes in the FITS
  standard, or to see its usage fall. In this paper we detail some
  selected important problems which exist within the FITS standard
  today.  It is not our intention to prescribe specific remedies to
  these issues; rather, we hope to call attention of the FITS and
  greater astronomical computing communities to these issues in the
  hopes that it will spur action to address them.
\end{abstract}

\section{Introduction}




\bibliography{P90}

\end{document}
